%%%%%%%%%%%%%%%%%%%%%%%%%%%%%%%%%%%%%%%%%
%  My documentation report
%  Objective: Write down many formulas and recipes that help me do my work
%
% Important note:
% Chapter heading images should have a 2:1 width:height ratio,
% e.g. 920px width and 460px height.
%
%
% The original template (the Legrand Orange Book Template) can be found here --> http://www.latextemplates.com/template/the-legrand-orange-book
%
% Original author of the Legrand Orange Book Template:
% Mathias Legrand (legrand.mathias@gmail.com) with modifications by:
% Vel (vel@latextemplates.com)
% Andrea Hidalgo from his book Clustering the Interestellar Medium
%
% Original License:
% CC BY-NC-SA 3.0 (http://creativecommons.org/licenses/by-nc-sa/3.0/)
%%%%%%%%%%%%%%%%%%%%%%%%%%%%%%%%%%%%%%%%%
 
%----------------------------------------------------------------------------------------
%	PACKAGES AND OTHER DOCUMENT CONFIGURATIONS
%----------------------------------------------------------------------------------------

\documentclass[11pt,fleqn]{book} % Default font size and left-justified equations

\usepackage[top=3cm,bottom=3cm,left=3.2cm,right=3.2cm,headsep=10pt,letterpaper]{geometry} % Page margins

\usepackage{xcolor} % Required for specifying colors by name
\definecolor{ocre}{RGB}{52,177,201} % Define the orange color used for highlighting throughout the book

% Font Settings
\usepackage{avant} % Use the Avantgarde font for headings
%\usepackage{times} % Use the Times font for headings
\usepackage{mathptmx} % Use the Adobe Times Roman as the default text font together with math symbols from the Sym­bol, Chancery and Com­puter Modern fonts

\usepackage{microtype} % Slightly tweak font spacing for aesthetics
\usepackage[utf8]{inputenc} % Required for including letters with accents
\usepackage[T1]{fontenc} % Use 8-bit encoding that has 256 glyphs

% Bibliography
\usepackage[style=alphabetic,sorting=nyt,sortcites=true,autopunct=true,babel=hyphen,hyperref=true,abbreviate=false,backref=true,backend=biber]{biblatex}
\addbibresource{bibliography.bib} % BibTeX bibliography file
\defbibheading{bibempty}{}

\input{structure} % Insert the commands.tex file which contains the majority of the structure behind the template

\begin{document}
\title{A Mathematical path to Engineering}

%----------------------------------------------------------------------------------------
%	TITLE PAGE
%----------------------------------------------------------------------------------------

\begingroup
\thispagestyle{empty}
\AddToShipoutPicture*{\put(0,0){\includegraphics[scale=1.25]{esahubble}}} % Image background
\centering
\vspace*{5cm}
\par\normalfont\fontsize{35}{35}\sffamily\selectfont
\textbf{A Mathematical path to Engineering}\\
{\LARGE A Constructive Decalogue of formulas and recipes}\par % Book title
\vspace*{1cm}
{\Huge Rodrigo Ramele}\par % Author name
\endgroup

%----------------------------------------------------------------------------------------
%	COPYRIGHT PAGE
%----------------------------------------------------------------------------------------

\newpage
~\vfill
\thispagestyle{empty}

%\noindent Copyright \copyright\ 2014 Andrea Hidalgo\\ % Copyright notice

\noindent \textsc{Summer Research Internship, University of Western Ontario}\\

\noindent \textsc{github.com/LaurethTeX/Clustering}\\ % URL

\noindent These notes have been gathered to 20 years of career.

\noindent \textit{First release, yet do not know} % Printing/edition date

%----------------------------------------------------------------------------------------
%	TABLE OF CONTENTS
%----------------------------------------------------------------------------------------

\chapterimage{head1.png} % Table of contents heading image

\pagestyle{empty} % No headers

\tableofcontents % Print the table of contents itself

%\cleardoublepage % Forces the first chapter to start on an odd page so it's on the right

\pagestyle{fancy} % Print headers again

%----------------------------------------------------------------------------------------
%	CHAPTER 1
%----------------------------------------------------------------------------------------

\chapterimage{head2.png} % Chapter heading image

\chapter{Introduction}

\section{Motivation}\index{Motivation}


Sometimes we get lost on the true origin of Engineering.
Which is the practical implementation of mathematics.



\section{Objective}\index{Objective}

The purpose of this work is to establish in a step by step and constructive fashion a set of mathematical concepts from a pragmatic point of view, of mathematical ideas that are useful to engineering.

Everything that is written here is exactly what I understand. If anything is besides this notes, I just put it in the category of "Not-understanding".  

What I found is that I came to this notes, many times, adding more information and connections to other things, and I will always understand the same things from a different and completely new perspective.

After a while I found that I will be very dissapointed of loosing those notes so I decided to write them down digitally. And have a backup with that.

How can you translate something that you created with your own pen ?  from your own handwriting ?  into something digital that is feel so cold to the naked eye.

Someone may found these notes useful.  I believe that will be for two reasons:  I wanted to create them in a constructive way: if I cannot find some rule to justify what I am doing then I am not doing anything, it doesn't exist.  The second reason is that I always found mathematics to be hard.  I love to understand it but it takes time to do it, to sit down.  At the same time, "undestanding" is not binary.  The best analogy I found is like a complex and evolutive graph or a tree-like structure, with many branches.  Let's say you understand something.  Then you have only "one path" to go from point A to point B of the concept.   But you can only walk across that path, and you, somehow, can follow the train of thinking from one place to the other.  However you cannot connect it to other things.  When you teach, somebody may point out the existence of point C closer to both.  But you were completely unaware of it until some student ask you something about it.  Then you discover that connection (and manage to give a response to the poor student, which it will be satisfactory or just mumble jamble).  Then you start to discover many more ramificiations and your level of understanding increases.  So understanding, is the level of ramification that you know of something, how much can you connect the concept to others.   

for Control engineerings for instance, everything can be a control problem.  The same can be applied to the "Rule of 3" or proportions or fractions.  Or algebra.  These very important tools allows you to tackle problems that is the reason they are so ubiquous.  In such way the Consciousness theory share some intuition with these concepts, measuring the level of integration and connectivity (Rhythms of the brain).



\subsection{References}\index{References}

Since I found so much good information about pretty much everything I wanted to know about, I will just create a remark and let you know where you can find more specific information about, just like below.

\begin{remark}
For more information about the cosmological principle, review Chapter 1: Why Learn Astronomy?, page 10, from \textbf{21st Century Astronomy}, \textit{Hester | Smith | Blumenthal | Kay | Voss}, Third Edition, 2010.
\end{remark}

%This statement requires citation \cite{book_key}; this one is more specific \cite[122]{article_key}.


%----------------------------------------------------------------------------------------
%	CHAPTER 2
%----------------------------------------------------------------------------------------
\chapterimage{band1.png}

\chapter{Set Theory: preliminaries}

%----------------------------------------------------------------------------------------
%	CHAPTER 3
%----------------------------------------------------------------------------------------

\chapterimage{boat.png}
\chapter{Precalculus}

%----------------------------------------------------------------------------------------
%	CHAPTER 4
%----------------------------------------------------------------------------------------

\chapterimage{head1.png} % Chapter heading image

\chapter{Signal Processing}


\section{Complex Numbers}

\begin{equation}
z = (x, y)
\label{eq:c1}
\end{equation}


\begin{equation}
Re \; z = x
\label{eq:c1}
\end{equation}


\begin{equation}
Im \; z = y
\label{eq:c1}
\end{equation}

\section{Further work}
Well, finally we reached the point where I my time in Canada finished and I this research is still on its first stages. I have so many ideas of how to explore the clustering techniques in the DAME platform, MatLab, Python and everything else that can be tested. 


\chapter{Number Theory}

\chapter{Pragmatic Control Theory}

That thing you are doing is proportional control.


\end{document}